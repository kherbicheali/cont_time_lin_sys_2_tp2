
\begin{appendices}

\chapter*{\textsc{Annexe A}}
	\addcontentsline{toc}{chapter}{\textsc{Annexe A}}		
	
	Code MATLAB qui calcule la matrice de commandabilité resp(observabilité) et son rang,\label{Annexe A} \hyperref[section 1.2]{Retour vers section 1.3}
	
	\begin{lstlisting}	
clear all;
close all;
clc;

Ke = 3.6/1000*60/(2*pi);
Ks=10;
Kg=0.105;
Te = 0.01;
Km=10;
Tm=0.3;
Kc=3.5/100;

A=[0 Ks/(9*Kg);0 (-1/Tm)]
B=[0; (Km*Kg)/Tm]
C=[1 0]
D=[0]

sys=ss(A,B,C,D)

Co=ctrb(sys);
rang_co=rank(Co);

Obs=obsv(sys);
rang_obs=rank(Obs);

	\end{lstlisting}	
	
	
\chapter*{\textsc{Annexe B}}
	\addcontentsline{toc}{chapter}{\textsc{Annexe B}}		
	
	Code MATLAB qui calcule la valeur du gain matriciel K,\label{annexe B} \hyperref[K]{Retour vers section 2.2}
	\begin{lstlisting}	
	
	K=acker(A,B,[-2.4+5.5*i -2.4-5.5*i])
	
	\end{lstlisting}	
	
\chapter*{\textsc{Annexe C}}
	\addcontentsline{toc}{chapter}{\textsc{Annexe C}}		
	
	Code MATLAB qui calcule la valeur du gain N,\label{Annexe C} \hyperref[N]{Retour vers section 2.2}
	\begin{lstlisting}	
	
	Abf=A-B*K
	Bbf=B;
	Cbf=C;
	Dbf=D;
	
	sysbf=ss(Abf,Bbf,Cbf,Dbf);

	dcg=dcgain(sysbf) 
	N=1/dcg
	
	\end{lstlisting}	
	
\end{appendices}	